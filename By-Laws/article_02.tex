{
\begin{article}{Suspension of Active Members}
	\item Suspended members are not permitted by the Executive Committee to take part in any activity associated with this organization.
	\item Members shall be suspended if they are placed on academic probation, or by unanimous vote of the Executive Committee.  Among these, any temporary officers shall immediately lose their positions, which may be reinstated at the discretion of the Executive Committee after the resolution of the suspension.  Upon the announcement of suspension, the organization shall be informed of the reason for suspension.
	\item If any officer is to be suspended he/she will be removed from his/her position immediately without following the impeachment process as outlined in \refConstitution{10}.
	\item If a suspension is resolved before the end of the current election cycle, the member in question shall not be eligible to take office and shall not take part in the Executive Committee until the next non-emergency election.
	\item The resolution of a suspension shall occur based upon its initial enactment:
	\begin{enumerate}
		\item Those suspended as a result of academic probation shall have their suspension revoked when they are removed from academic probation.
		\item Those suspended by any past or present Executive Committee may have their suspension lifted by unanimous vote of the current Executive Committee.  If a contested failure to resolve this suspension occurs, in which a non-unanimous majority of the Executive Committee votes to resolve the suspension, it will be put to a vote of a quorum.
	\end{enumerate}
	When a resolution is decided among the Executive Committee, the membership will be informed by the Executive Committee at the next weekly meeting.  The resolution is not finalized until the membership agrees on the resolution by way of Section 6 of \refByLaws{2}.
	\item If a decision to enact, resolve, or maintain a suspension made by the Executive Committee occurs, but is contested by any member not in the Executive Committee, the reinstatement or revoking of the suspension shall be put to a vote of a quorum.
\end{article}
}