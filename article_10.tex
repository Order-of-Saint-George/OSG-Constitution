{
\begin{article}{Impeachment and Removal of Officers}
	\item Any officer in the organization may be removed from office for offenses including, but not necessarily limited to, gross neglect of duty, misuse of official position, or failure to abide by the Grove City College Standards of Conduct as defined in \crimson.
	\item In the event that a member has charges to bring against an officer, the organization shall follow the impeachment process described in Section 3 of Article X, below, to determine whether the officer is guilty of the charges. The matter shall be resolved internally between the Executive Committee and the Advisor. Upon a three-quarter majority vote of the officers not being charged, the officer in question shall be removed from the organization.
	\item The impeachment process contains four distinct steps: declaration of charges, review of charges, executive hearing, and vote of no confidence.  In addition, no attorneys shall be present during any of the impeachment proceedings stated below.

	\hypertarget{Declaration of Charges}{}
	\bookmark[dest=Declaration of Charges, level=4]{Declaration of Charges}
	\textbf{Declaration of Charges}

	The member of the organization who has a charge to bring against an officer and therefore wishes to initiate the impeachment process shall make his intentions known to the person charged before submitting in writing a charge to the President, or if he is the one being charged, to the Vice President. The officer receiving the charge shall acknowledge receipt in writing. The Advisor shall be notified and shall function in an advisory capacity to the President (or Vice President, if applicable).

	\hypertarget{Officer Review of Charges}{}
	\bookmark[dest=Officer Review of Charges, level=4]{Officer Review of Charges}
	\textbf{Officer Review of Charges}

	The President (or Vice President, if applicable) and the member bringing forth the charge shall then speak to the charged officer. If this officer refuses to step down from his position and the problem is not resolved at this stage, the process shall move on to the hearing of charges. The President (or Vice President, if applicable) shall have the authority to dismiss charges if he thinks they are not legitimate.

	\hypertarget{Executive Hearing}{}
	\bookmark[dest=Executive Hearing, level=4]{Executive Hearing}
	\textbf{Executive Hearing}

	The President (or Vice President, if applicable) shall then call a meeting of the Executive Committee to hear and review the matter. The Executive Hearing shall be held no sooner than one week and no later than three weeks after the charge has been made. The member bringing the charge against the officer shall be invited to present his case to this group, and the officer being charged shall be invited to respond to the charges.

	\hypertarget{Vote of No Confidence}{}
	\bookmark[dest=Vote of No Confidence, level=4]{Vote of No Confidence}
	\textbf{Vote of No Confidence}

	Upon conclusion of the hearing, the officers not being charged shall review the matter and vote as to whether removal from position occurs; a three-quarter majority of these officers is required to remove the accused from office. If this majority is not reached, or if the officers are able to resolve the matter at this hearing, the impeachment process shall be considered \enquote{resolved} and shall conclude at this step. If any officer is to be relieved of his position, elections shall be held as soon as possible to fill the vacancy according to Section 7 of \refConstitution{5}.
\end{article}
}